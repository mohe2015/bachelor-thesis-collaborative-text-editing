\chapter{Future Work} \label{chapter:future-work}

In this chapter we look at what is missing and which aspects could be researched further.

\paragraph{Investigating OT Algorithms}

In their review of the Fugue paper, \Citeauthor*{2023-sun-critical-examination-fugue-ot} shows that the claims in the Fugue paper \cite{2023-weidner-minimizing-interleaving} about \gls{ot} being interleaving are not correct \cite{2023-sun-critical-examination-fugue-ot,
  2023-sun-critical-examination-fugue-ot-1,2023-sun-critical-examination-fugue-ot-2,2023-sun-critical-examination-fugue-ot-3}.
First, they show that mistakes were made in the Fugue paper when applying the \gls{ot} algorithms which render their results regarding \gls{ot} invalid \cite{2023-sun-critical-examination-fugue-ot-1,2023-sun-critical-examination-fugue-ot-2}.
They also show that interleaving has been examined and documented before and can be solved in \gls{ot}, usually by having operations based on strings and not single characters, but this is also possible when operating on single characters \cite{2023-sun-critical-examination-fugue-ot-2}. Therefore, investigating OT algorithms, especially in a \textit{non-realtime} setting could be interesting.

\paragraph{Necessary Non-interleaving Properties for Intent-Preserving Text Editing}

The review by \Citeauthor*{2023-sun-critical-examination-fugue-ot} also suggests that not all the properties that are proposed in the Fugue paper (especially multi-user relay interleaving and backward interleaving) are necessary or useful for user intent preserving text editing \cite{2023-sun-critical-examination-fugue-ot-2}. While the examples we show in \Cref{section:challenges-text-interleaving} and \Cref{section:challenges-text-interleaving-fugue} are realistic, we do not know which properties are strictly necessary, as required properties seriously limit the freedom in the design of suitable algorithms. For example, the Fugue paper proposes a property of maximally non-interleaving that produces a unique order with the least possible interleaving. While it is interesting that this property produces a unique order it is unclear whether this is useful in practice. Our example in \Cref{section:challenges-text-interleaving-fugue} also shows that this property is not sufficient for non-interleaving when deletions are involved. Future work could investigate how this property could be adapted to better model non-interleaving in such cases.

\paragraph{Privacy} \label{sec:data-privacy-issues}

One big problem we see with all these algorithms is that it is hard or impossible to properly delete data in case a user wishes to do so while still being able to converge and preserve user intentions. Future work could investigate which possibilities exist to actually remove deleted text. One possibility could be to clear the deleted characters in the tree. This would also work for the causal broadcast messages but then undo would not be possible anymore. Therefore, maybe more control is needed for end users whether they want to do a normal deletion or a permanent deletion, which would break undo, and also to see which data is still visible in the internal data structure or in the message log.

As the messages are only required to be processed by the peers themselves, adding encryption should be comparatively easy. This could also route messages over a server and store them there without the server being able to read the contents. Some thought should still be put into what can be inferred from metadata like message timing and size. For example, it would likely be possible for the server to find out which user writes how many characters at what time.

\paragraph{Correctness} \label{section:future-work-correctness}

Currently, there is little protection against messages that do not conform to the expected rules. For example if two peers send different characters with the same ID this will create inconsistencies or potentially also crashes. Also peers can easily send characters for other peers as the peer ID is not verified to be only used by the respective peer. This should be tested more, for example using fuzzing tests that can send arbitrary messages that do not conform to the rules. Also, inconsistencies by different characters with the same ID should be avoided, for example by making the character part of the ID.

While our property tests seemed to find all relevant issues, they were pretty limited for tests with multiple replicas and could not check the exact expected outcome in that case. Therefore, it may be interesting to find ways to more thoroughly test this while also testing non-interleaving.

\paragraph{Usability} \label{future-work-usability}

Rich text is probably the largest missing feature that may also lead to many design challenges. First, there is inline formatting like bold, underlined, italic, strike-through, subscript or superscript text. But there is also structural formatting like headings, subheadings, ordered and unordered lists, tables, etc. Both create new challenges with user intent. While for \gls{ot} there is a lot of previous work which is also successfully used in production e.g. Google Docs\footnote{\url{https://www.google.com/docs/about/}}, for \glspl{crdt} there is not much previous research \cite{2022-litt-peritext}. The Peritext paper \cite{2022-litt-peritext} investigates inline formatting and shows some problems in prior algorithms with correctly preserving user intentions \cite{2022-litt-peritext}. For example the Yjs algorithm based on \gls{yata} \cite{2016-yata-yjs} adds markers where inline formatting starts and where it ends into the text. This fails to handle a simple case where a bold text is unbolded and concurrently part of that bold text is unbolded which then leads to unrelated text getting bold \cite[Section 2.3.2]{2022-litt-peritext}.

In a collaborative context it needs to be possible to undo arbitrary actions by any user and not only the last action like it is usually the case in traditional editors. Therefore, support for so-called selective undo is needed. For \gls{ot} algorithms, transformations need to be applied to the correct document context \cite{2009-sun-ot-context-undo}. This means the control algorithms need to properly handle this and transformation functions potentially need to uphold specific properties \cite{2009-sun-ot-context-undo}.

When part of a text is moved and concurrently part of that text is edited it would make sense that these edits are correctly preserved. As normal copy and paste does not track this state this needs a special operation or needs to store the necessary metadata in the clipboard. Also, this needs support at the \gls{crdt} level \cite{2022-anjana-move,2023-kleppmann-json-move}.

Instead of operating on a character level it could make sense to operate on a string level. This would be more efficient and could have better semantics for range deletions, copy and paste or moving text. For \gls{ot} this seems to often be done but is much more complicated, especially in combination with undo \cite{2024-sun-ot-faq}.

While we did not look at this in this thesis, it is not hard to serialize and deserialize our representation. It may be interesting to find out which parts of the data structures, that are only needed to improve lookup performance, should be persisted to storage and which parts can be quickly rebuilt on loading.

\paragraph{Performance} \label{section:future-work-performance}

While in some cases the full editing history needs to be kept to be able to attribute all changes, in other cases it can be reduced as much as possible without causing causality problems. The approach of the antimatter\footnote{\url{https://web.archive.org/web/20240623153539/https://braid.org/antimatter}} algorithm is to combine operations that have been seen by the same group of peers by tracking acknowledgements. In case peers go offline but come online at some point later it can potentially still combine operations.

Our current performance measurements only test non-concurrent actions. It may be beneficial to either find or create some real-world editing trace with concurrent actions or generate some artificial trace like in YATA \cite[Section 6.1]{2016-yata-yjs}.

The memory usage per character is pretty high, even for the real world benchmark. Except for using a low-level language it could also make sense to investigate how to only create the cache for the leftmost and rightmost descendant if they are deeply nested which based on \Cref{sec:memory-results} would likely save large amounts of memory.

When the data is larger than the available memory, our algorithm currently can only be used with swapping. Future work could look into alternatives, for example to store currently not edited parts to disk.
