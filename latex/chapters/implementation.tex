\chapter{Implementation of Fugue Algorithm} \label{section:implementation}

This chapter first lists the requirements for an implementation of the Fugue algorithm in \Cref{section:implementation-requirements}. \Cref{section:implementation-browser} gives insights into our editor implementation in the browser and our \gls{p2p} functionality and \Cref{sec:synchronization} explains how our synchronization works and is optimized. \Cref{sec:property-tests} introduces our use of property tests to ensure convergence of our implementation and argues that extensive use of assertions for invariants aids in finding the root cause of test failures. Finally, \Cref{section:implementation-issues-algorithmic-description} reveals some small issues in the algorithmic description in the Fugue paper \cite{2023-weidner-minimizing-interleaving}.

The source code is available at:\\
\url{https://github.com/mohe2015/bachelor-thesis-collaborative-text-editing}

\section{Required Implementation Functionality} \label{section:implementation-requirements}

Based on the Fugue paper \cite{2023-weidner-minimizing-interleaving} and our explanation of the Fugue algorithm in the last chapter, an implementation needs to provide the following functionality:

It needs to provide an interface to a tree with left and right children, possibly multiple children on each side but usually only one on one side. It requires fast retrieval of a node based on an index in the non-deleted node traversal and fast retrieval of an index in the non-deleted node traversal based on the node. Additionally, it requires fast retrieval of a node based on its ID. For initial loading it also needs to be able to traverse the whole tree in order. Furthermore, inserting nodes to the right and left of other nodes needs to be efficient with the special case of multiple left or right children.

In practice, trees usually contain many deep right descendants because of consecutive character insertions \cite[Figure 5]{2023-weidner-minimizing-interleaving}, so this is a case that should be heavily optimized.

\begin{listing}
  \begin{minted}{scala}
val schema = Schema(SchemaSpec(orderedmap.from(StringDictionary(
  ("text", NodeSpec()),
  ("doc",
    NodeSpec()
      .setContent("text*")
      .setMarks("")
      .setCode(true)
      .setDefining(true)
      .setParseDOM(
        js.Array(TagParseRule("pre").setPreserveWhitespace(full)))
      .setToDOM(_ => Array("pre", 0)))))))
val hardBreakCommand: Command = (state, dispatch, view) => {
  dispatch.get(state.tr.insertText("\n"))
  true
}
val editorStateConfig = EditorStateConfig().setSchema(schema)
  .setPluginsVarargs(keymap(StringDictionary(("Enter", hardBreakCommand))))
\end{minted}
  \caption{Code excerpt of ProseMirror schema setup}
  \label{lst:prosemirror-schema}
\end{listing}



\section{Browser Implementation of Text Editor} \label{section:implementation-browser}

To properly use text editing algorithms an editor is required, so we implement an interface to ProseMirror\footnote{\url{https://prosemirror.net/}} and transpile Scala to JavaScript using Scala.js\footnote{\url{https://www.scala-js.org/}} to be able to use our implementation on the web.

By default, ProseMirror creates newlines using \texttt{<br/>}~tags and paragraphs using \texttt{<p>}~tags. This makes it complicated to convert between the ProseMirror document offset and the text offset. Therefore, we configured ProseMirror to only support plaintext and use \texttt{\textbackslash n} for newlines and configured the browser to render \texttt{\textbackslash n} as newlines (which does not work by default) as shown in \Cref{lst:prosemirror-schema}.

We also implemented a demo using WebRTC\footnote{\url{https://webrtc.org/}} to collaboratively edit a text. It keeps the full history on all connected peers, so it is not possible to permanently delete anything. This is the reason for not implementing persistence, see \Cref{sec:data-privacy-issues}.

\section{Synchronization of Changes} \label{sec:synchronization}

The changes are synchronized using causal broadcast as in the Fugue paper \cite{2023-weidner-minimizing-interleaving}. The events are ordered using vector clocks \cite{1988-mattern-vector-clock,1988-fidge-vector-clock}. Only change synchronization updates the vector clock. Therefore, the clock does not need to be updated while working offline, and the changes can be sent in one batch which is more efficient. Instead of creating a message per character insertion or deletion, consecutive deletions and insertions that have the same causality are combined to optimize memory usage.

\section{Testing Using Property Tests}\label{sec:property-tests}

Property tests are a core part of testing \glspl{rdt} as the existence of numerous edge cases make unit testing infeasible. The tests run both on the internal data structure, with an interface for inserting and deleting characters at indices, and on the local web application as a Playwright\footnote{\url{https://playwright.dev/java/}} test.

The property tests run using ScalaCheck\footnote{\url{https://scalacheck.org/}} and specifically its stateful testing support\footnote{\url{https://github.com/typelevel/scalacheck/blob/main/doc/UserGuide.md\#stateful-testing}} using \texttt{Commands}\footnote{\label{footnote:commands}\url{https://github.com/typelevel/scalacheck/blob/main/core/shared/src/main/scala/org/scalacheck/commands/Commands.scala}}. ScalaCheck \texttt{Commands} store a system under test and a state that is compared to the system under test. Possible actions are defined by implementing the \texttt{Command}\footref{footnote:commands} trait. The trait has several methods for pre conditions, post conditions, running the action and calculating the next state. ScalaCheck generates \texttt{Command}s and their contents using Generators, e.g. \texttt{Gen.chooseNum(0, Int.MaxValue)} which are then run by ScalaCheck against the system under test and if failures occur it tries to simplify the failure case.

Our property tests randomly create replicas, synchronize replicas, insert text at a replica or delete text at a replica. Then, they check whether replicas have the same text after they synchronized. Unfortunately it is not easily possible to check \textit{what} the expected text would be as that would need more or less a reimplementation of the synchronization logic, see \Cref{section:future-work-correctness}. We also have property tests that check that local operations match the same operations on a \texttt{String}.

While trying out new approaches, implementation mistakes are likely, particularly when more complicated approaches have lots of edge cases. It is really laborious to find the root cause for every test failure to fix edge cases, especially for property tests that do not always produce the smallest possible test case. It helps significantly to add lots of assertions into the code that not only check local conditions like traditional uses of assertions but also check global invariants. Some examples of such assertions are ensuring that parent and child references are symmetric to each other and that insertions and deletions correctly update the positions of all characters. These assertions strongly affect the performance, so they need to be disabled for production use.

Ideally, invariant assertions would be automatically checked after every object creation and modification, but that is not easily possible with Scala. Therefore, they were added manually at relevant places. The tests also detect the bugs without these invariant assertions. The failure then happens at a later time in execution, which complicates finding the root cause, but does not decrease the reliability.

\section{Issues in the Algorithmic Description} \label{section:implementation-issues-algorithmic-description}

While working on our implementation, we found that the algorithmic description \cite[Algorithm~1]{2023-weidner-minimizing-interleaving} is, for the most part, satisfactory. However, it contains one large issue. While the Fugue paper includes the conversion from character offsets to their internal representation, it misses the reverse direction \cite[Algorithm~1]{2023-weidner-minimizing-interleaving}. Received operations also need to be converted to the index to update the local text editor. Therefore, we extended the algorithmic description with that. This is not just relevant for implementation but also for optimization, which we address in the next chapter. It means that further functionality is required, that can map a node ID to the position in the tree traversal of visible nodes, which is the visible text. The remaining issues were only minor or instances of suboptimal specification.

First, the ID type \cite[Algorithm~1]{2023-weidner-minimizing-interleaving} can always be \texttt{null}. As this can only be the case for the root node, we moved this case to the places where the root node could potentially be used. There are some places where this could \textit{not} be the case, e.g. remote insertions can not send the root node as the root node is always locally created.

Second, in line 10 of the description \cite[Algorithm~1]{2023-weidner-minimizing-interleaving}, root is initialized with a value that is invalid according to their specification because the side can only be \texttt{L} or \texttt{R} but never \texttt{null} according to the types. Our implementation arbitrarily chooses the root node to be on the right side to simplify checks at other places in the code. An alternative would be to use an enumeration for the node and not have an ID, value, side and parent for the root node at all.

Third, each node does not necessarily need to store the ID of its parent and children \cite[Algorithm~1]{2023-weidner-minimizing-interleaving}. It could also store a reference directly to them.

Lastly, the node after \texttt{leftOrigin} in line 24 \cite[Algorithm~1]{2023-weidner-minimizing-interleaving} can be retrieved as the leftmost descendant of the first right child of the \texttt{leftOrigin}. The leftmost descendant is the node that is reached by repeatedly descending into the leftmost child until there are no left children. This is logical as the next node must be in the right subtree and there the first node is the leftmost node. Depending on the implementation that may be faster or easier.
